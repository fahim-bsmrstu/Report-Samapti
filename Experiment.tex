\chapter{Experimental Result \& Discussion}
\label{Ch_Chapter5}


\section{Experimental Result}

To examine our ranking, we collect 30 bangla documents from the bangla daily newspaper eg Prothom alo, kaler kontho etc. The documents are written and saved in the text files using UTF-8 format. For each document in our corpus, we consider only one human ranking for evaluation. Evaluation of a system produced ranking is done by comparing it to the human ranking. There are some documents here on boimela.

Doc 1 : boimela txt.\\
কলকাতায় বাংলাদেশ বইমেলা থাকছে ৫০ প্রকাশনী । বইয়ের বন্ধুত্ব সীমানা ছাড়িয়ে- স্লোগানে ১ সেপ্টেম্বর থেকে কলকাতায় শুরু হচ্ছে ১০ দিনব্যাপী ‘বাংলাদেশ বইমেলা’। সচিবালয়ে সোমবার সংস্কৃতি সচিব আকতারী মমতাজ এক সংবাদ সম্মেলনে জানান, ষষ্ঠবারের মতো আয়োজিত মেলায় বাংলাদেশের ৫০টি প্রকাশনা প্রতিষ্ঠান অংশ নেবে। ১ সেপ্টেম্বর বিকেল ৫টায় মেলার উদ্বোধন করবেন ইমেরিটাস অধ্যাপক আনিসুজ্জামান। উদ্বোধনী অনুষ্ঠানে থাকবেন পশ্চিমবঙ্গ সরকারের শিক্ষামন্ত্রী পার্থ চট্টোপাধ্যায় ও কবি-প্রাবন্ধিক শঙ্খ ঘোষ। প্রতিদিন দুপুর ২টা থেকে রাত ৮টা পর্যন্ত মেলা উন্মুক্ত থাকবে। শনি ও রোববার বিকাল ৩টা থেকে রাত ৮টা পর্যন্ত মেলা চলবে।
জাতীয় গ্রন্থকেন্দ্র ও রপ্তানী উন্নয়ন ব্যুরোর সহযোগিতায় এবং কলকাতায় বাংলাদেশ উপ-দূতাবাসের ব্যবস্থাপনায় বাংলাদেশ জ্ঞান ও সৃজনশীল প্রকাশক সমিতি গত পাঁচ বছর ধরে কলকাতায় বাংলাদেশ বইমেলার আয়োজন করছে।প্রথম তিন বছর মেলাটি গণকেন্দ্র শিল্প সংগ্রহশালায় হলেও গত দুই বছর ধরে রবীন্দ্র সদনের উন্মুক্ত প্রাঙ্গণে হয়। এবারও এই উন্মুক্ত প্রাঙ্গণে বাংলাদেশ বইমেলা বলে জানান আকতারী মমতাজ।

