\chapter{Related Works}
\label{Ch_Chapter2}
%
%\section{Introduction}
%

%\section{C এর কম্পাইলার}
%আমরা আগের Chapter এ জেনেছি যে, মানুষের ভাষার কাছাকাছি লিখিত ভাষাকে যন্ত্রের কাছে বোধগম্য 
%করে তুলে একটি software দোভাষী যাকে বলা হয় compiler ।
%
%C একটি structured প্রোগ্রামিং ল্যাংগুয়েজ। এর প্রতিটি পর্যায়ে নির্ভুলভাবে নিয়ম মেনে চলতে হয়। এমনকি
%C প্রোগ্রামিং ল্যাংগুয়েজটি ছোট হাতের অক্ষর এবং বড় হাতের অক্ষরকে আলাদা ভাবে দেখে, case sensitive ।
%এই নিয়মের অন্যথা হলে C এর জন্য নির্ধারিত compiler টি এই ভাষাকে আর যন্ত্রের ভাষাতে 
%(machine language) রূপান্তর করতে পারে না।